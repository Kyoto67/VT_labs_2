\section{1322. Шпион}
\textit{Условие задачи:} \par
Спецслужбы обнаружили действующего иностранного агента. Шпиона то есть. Установили наблюдение и выяснили, что каждую
неделю он через Интернет посылает кому-то странные нечитаемые тексты. Чтобы выяснить, к какой информации получил доступ
шпион, требуется расшифровать информацию. Сотрудники спецслужб проникли в квартиру разведчика, изучили шифрующее устройство
и выяснили принцип его работы. На вход устройства подается строка текста $S_1$ = $s_1$$s_2$...$s_N$. Получив ее, устройство
строит все циклические перестановки этой строки, то есть $S_2$ = $s_2$$s_3$...$s_N$$s_1$, ..., $S_N$ = $s_N$$s_1$$s_2$...$s_{N-1}$.
Затем множество строк $S_1$, $S_2$, ..., $S_N$ сортируется лексикографически по возрастанию. И в этом порядке строчки
выписываются в столбец, одна под другой. Получается таблица размером N $\times$ N. В какой-то строке K этой таблицы
находится исходное слово. Номер этой строки вместе с последним столбцом устройство и выдает на выход.\\
Например, если исходное слово $S_1$ = \textit{abracadabra}, то таблица имеет такой вид:
\begin{enumerate}
    \item \textit{aabracadabr} = $S_11$
    \item \textit{abraabracad} = $S_8$
    \item \textit{abracadabra} = $S_1$
    \item \textit{acadabraabr} = $S_4$
    \item \textit{adabraabrac} = $S_6$
    \item \textit{braabracada} = $S_9$
    \item \textit{bracadabraa} = $S_2$
    \item \textit{cadabraabra} = $S_5$
    \item \textit{dabraabraca} = $S_7$
    \item \textit{raabracadab} = $S_10$
    \item \textit{racadabraab} = $S_3$
\end{enumerate}
И результатом работы устройства является число 3 и строка \textit{rdarcaaaabb}.
Это все, что известно про шифрующее устройство. А вот дешифрующего устройства не нашли. Но поскольку заведомо известно,
что декодировать информацию можно (а иначе зачем же ее передавать?), Вам предложили помочь в борьбе с хищениями секретов
и придумать алгоритм для дешифровки сообщений. А заодно и реализовать дешифратор.
\textit{Пояснение к примененному алгоритму:} \par
\BgThispage
\newpage
\textit{Код:}
\small
\begin{center}
    \begin{verbatim}
    \end{verbatim}
\end{center}
\normalsize
\BgThispage
\newpage


\section{1604. В Стране Дураков}
\textit{Условие задачи:} \par
\textit{Пояснение к примененному алгоритму:} \par
\textit{Код:}
\small
\begin{center}
    \begin{verbatim}
    \end{verbatim}
\end{center}
\normalsize
\BgThispage
\newpage
