\thispagestyle{empty}
\BgThispage
\section{Описание метода} \par
Суть метода заключается в последовательном алгоритме, проходящему по диагонали:
\begin{enumerate}
    \item Выбор главного элемента - максимальный коэффициент в стобце текущего элемента, между строкой с текущим элементом и располагающимися ниже.
    \item Свап текущей строки и строки с выбранным текущим главным элементом
    \item Последовательное вычитание из строк, располагающихся ниже, текущей строки, умноженной на частное коэффициента элемента, располагающегося в столбце текущего главного элемента в вычитаемой строке, и главного элемента.
\end{enumerate}
После прохождения алгоритмом, описанным выше, до конца диагонали, получим треугольную матрицу, из которой необходимо сделать диагональную обратным алгоритмом -- последовательно вычитаем каждую строку из строк, располагающихся выше, умноженную на коэффициент множителя из этой строки, располагающегося в столбце с элементом диагонали (главным элементом, выбранным ранее).
Таким образом, получим диагональную матрицу, из которой можно получить множество решений $X$, где для каждого $x_{i}$ = $\frac{B_{i}}{A_{i,i}}$.
\newpage
\thispagestyle{empty}
\BgThispage
\section{Блок-схема численного метода}
skip\\
\newpage
\thispagestyle{empty}
\BgThispage
\section{Listing реализованного численного метода}
\tiny
\begin{verbatim}
    void toTriangleForm() throws MatrixCreateException {
        double[][] A = matrix.getA();
        double[] B = matrix.getB();
        int n = matrix.getSize();
        for (int i = 0; i < n; i++) {
            replaceLinesInMatrix(A, B, i, chooseMainEl(A, i));
            lineNormalization(A[i], A[i][i], B, i);
            for (int j = i + 1; j < n; j++) {
                lineSub(A[j], A[i], B, j, i, A[j][i]);
            }
        }
        this.matrix = new Matrix(A, B, n);
    }

    void fromTriangleToDiag() throws MatrixCreateException {
            double[][] A = matrix.getA();
            double[] B = matrix.getB();
            for (int i = matrix.getSize() - 1; i >= 0; i--) {
                for (int j= i -1; j>=0 ; j--) {
                    lineSub(A[j], A[i], B, j, i, A[j][i]);
                }
            }
            matrix = new Matrix(A, B, matrix.getSize());
        this.myMatrixIsDiagonal = true;
    }

    private void lineNormalization(double[] line, double a, double[] B, int lineIndex) {
        if (a == 0 || a == 1) return;
        for (int i = 0; i < line.length; i++) {
            line[i] /= a;
        }
        B[lineIndex] /= a;
    }

    private void lineSub(double[] line2, double[] line1, double[] B, int lineIndex2, int lineIndex1, double multiplier) {
        for (int i=0; i<line2.length; i++) {
            line2[i] -= line1[i]*multiplier;
        }
        B[lineIndex2] -= B[lineIndex1]*multiplier;
    }

    private void replaceLinesInMatrix(double[][] A, double[] B, int n1, int n2) {
        double b = B[n1];
        double[] a = A[n1];
        B[n1] = B[n2];
        B[n1] = b;
        A[n1] = A[n2];
        A[n2] = a;
    }

    private int chooseMainEl(double[][] A, int j) {
        int numberOfMainElLine = j;
        double maxEl = A[j][j];
        for (int i=j; i<A.length; i++) {
            if (A[i][j] > maxEl) {
                maxEl = A[i][j];
                numberOfMainElLine = i;
            }
        }
        return numberOfMainElLine;
    }
\end{verbatim}
\normalsize
\newpage
\thispagestyle{empty}
\BgThispage
\section{Examples}
\tiny
\begin{verbatim}
Enter the coefficients of the equation? (otherwise will be generated automatically) Y/N: Y
Enter the dimensionality of the matrix: 5
Enter the coefficients of matrices A and B line by line, separating them with a space:
0 0 0 0 0 1
0 0 0 0 0 2
0 0 0 0 0 3
0 0 0 0 0 4
0 0 0 0 0 5
Input completed.
Diagonalized matrix:
		0.0 0.0 0.0 0.0 0.0 | 1.0
		0.0 0.0 0.0 0.0 0.0 | 2.0
		0.0 0.0 0.0 0.0 0.0 | 3.0
		0.0 0.0 0.0 0.0 0.0 | 4.0
		0.0 0.0 0.0 0.0 0.0 | 5.0

This matrix has no solutions!
\end{verbatim}
\normalsize
Вывод верен - матрица действительно не имеет решений\\
\tiny
\begin{verbatim}
Enter the coefficients of the equation? (otherwise will be generated automatically) Y/N: Y
Enter the dimensionality of the matrix: 5
Enter the coefficients of matrices A and B line by line, separating them with a space:
1 1 1 1 1 1
1 1 1 1 1 2
1 1 1 1 1 3
1 1 1 1 1 4
1 1 1 1 1 5
Input completed.
Diagonalized matrix:
		1.0 1.0 1.0 1.0 1.0 | 1.0
		0.0 0.0 0.0 0.0 0.0 | 1.0
		0.0 0.0 0.0 0.0 0.0 | 2.0
		0.0 0.0 0.0 0.0 0.0 | 3.0
		0.0 0.0 0.0 0.0 0.0 | 4.0

This matrix has no solutions!
\end{verbatim}
\normalsize
Вывод верен - матрица действительно не имеет решений\\
\tiny
\begin{verbatim}
Enter the coefficients of the equation? (otherwise will be generated automatically) Y/N: Y
Enter the dimensionality of the matrix: 5
5.5 2.3 6.7 5.3 4.8 4.2
5.68 5.12 9 8.56 7.432 5
3 5 3 2 1 6.23
5 0 0 4 2.1 1
1 1 1 4 5 7
Enter the coefficients of matrices A and B line by line, separating them with a space:
Input completed.
Diagonalized matrix:
		5.68 0.0 0.0 0.0 0.0 | -120.120633
		0.0 2.295775 0.0 0.0 0.0 | 8.822536
		0.0 0.0 -0.509202 0.0 0.0 | 8.262291
		0.0 -0.0 0.0 -26.569904 0.0 | -1952.403274
		0.0 0.0 0.0 -0.0 -4.961458 | 252.562687

Determinant: -875.3204265714517

Residual:
		r1 = 1.4210854715202004E-14
		r2 = 0.0
		r3 = 0.0
		r4 = 0.0
		r5 = 0.0

Answer: x1 = -21.147998767605635
		x2 = 3.8429445394256843
		x3 = -16.22595944242167
		x4 = 73.4817586845628
		x5 = -50.90493298542485
\end{verbatim}
\normalsize
Детерминант, корни матрицы и невязка вычислены верно\\
\newpage
\thispagestyle{empty}
\BgThispage
\section{Вывод}